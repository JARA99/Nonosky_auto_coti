\documentclass[letterpaper]{article}

\usepackage[spanish,mexico]{babel}
\usepackage[svgnames,table]{xcolor}
\usepackage{graphicx}

\usepackage{geometry}
\geometry{ 
left=20mm,
right=20mm,
top=5mm,
bottom=5mm,
includehead,
includefoot,
headsep = 15mm,
footskip = 30mm
}

\usepackage{tikz}
\usepackage{pdfpages}

\usepackage{lipsum}

\usepackage{fontspec}
\setmainfont[
Path=fonts/,
Extension=.ttf,
UprightFont=*-Regular,
BoldFont=*-Bold,
Scale=1.2,
]{Roboto}



\newcommand*{\arraycolor}[1]{\protect\leavevmode\color{#1}}
% \newcolumntype{A}{>{\columncolor{blue!50!white}}c}
% \newcolumntype{B}{>{\columncolor{LightGoldenrod}}c}
% \newcolumntype{C}{>{\columncolor{FireBrick!50}}c}
\newcolumntype{g}{>{\columncolor{Gray!40}}c}
\newcolumntype{b}{>{\columncolor{Black}\color{White}}c}
\newcolumntype{G}{>{\columncolor{Gray!40}}m}
\newcolumntype{B}{>{\columncolor{Black}\color{White}}m}
\newcolumntype{C}{>{\centering}m}


\newcommand{\bgimage}[1]{
	\begin{tikzpicture}[remember picture, overlay]
	    \node[anchor=center, inner sep=0pt, outer sep=0pt, at=(current page.center)] {
		\includegraphics[width=1.01\paperwidth]{#1}
	    };
	\end{tikzpicture}
}

\newcommand{\descriptionbox}[2]{
	\begin{tikzpicture}[remember picture, overlay]
	    \node[anchor=center,
	    xshift=64mm, yshift=34mm,
	    inner sep=0pt,
	    at=(current page.south west),
	    text width = 10cm] {
		\begin{tabular}{lp{75mm}}
			% \hline
			\textbf{Empresa:} & #1\\
			\textbf{Contacto:} & #2\\
			% \hline
		\end{tabular}
	    };
	\end{tikzpicture}
	}

\newcommand{\cottab}[4]{
	\arrayrulewidth=1pt
	\renewcommand{\arraystretch}{1.5}
	\rowcolors{3}{.!80!Black}{}
	\begin{tabular}{bg}
	  \multicolumn{2}{b}{\bfseries Cotización No. #1}\\
		Contacto & #2\\
		Fecha & #3\\
		Cliente & #4\\
	\end{tabular}
	}

\newcommand{\desctab}[3]{
	\arrayrulewidth=1pt
	\renewcommand{\arraystretch}{1.5}
	\rowcolors{3}{.!80!Black}{}
	\begin{tabular}{bg}
	  \multicolumn{2}{b}{\bfseries Descripción de pago y entrega}\\
		Tipo de pago & #1\\
		Días de crédito & #2 \\
		Tiempo de entrega & #3\\
	\end{tabular}
	}


\begin{document}
	\thispagestyle{empty}
\bgimage{bgs/Portada.pdf}
\descriptionbox{Nonosky}{Danilo Gonzalez}

\pagebreak

\thispagestyle{empty}
\bgimage{bgs/Portada2.pdf}
\empty

\pagebreak

\begin{flushright}

\cottab{0001}{Danilo Gonzalez}{\today}{Nonosky}

\vspace{10pt}

\desctab{Contado, cheque o transferencia}{30 días de crédito.}{1 mes aproximadamente.}

\vspace{10pt}

\rowcolors{3}{}{}

\begin{tabular}{|p{0.43\linewidth}|C{0.15\linewidth}|C{0.15\linewidth}|C{0.15\linewidth}|}
    % \multicolumn{1}{H{0.5\linewidth}}{Descripción de proyecto A} & \multicolumn{1}{H{0.15\linewidth}}{Precio por unidad} & Cantidad & Total \\
    \rowcolor{gray} Descripción de proyecto & Precio por unidad & Cantidad & Total\\
    Instalación A & 1000.000000 & 3.000000 & 3000 \\
    Instalación B & 300.000000 & 40.000000 & 12000 \\
    Equipo C & 1200.000000 & 3.000000 & 3600 \\
    Total & NaN & NaN & 18600 \\
\end{tabular}
    \hspace{1pt}

\end{flushright}

\end{document}
